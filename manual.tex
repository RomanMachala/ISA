\documentclass[11pt, a4paper, hidelinks]{article}[08.10.2023]
    \usepackage[left=1.4cm, top=2.3cm, text={18.2cm, 25.2cm}]{geometry}
    \usepackage[utf8]{inputenc}
    \usepackage[czech]{babel}
    \usepackage[IL2]{fontenc}
    \usepackage{times}
    \usepackage{hyperref}
    \usepackage{listings}
    \usepackage{xcolor}

\lstdefinestyle{CStyle}{
    language=C,
    basicstyle=\ttfamily\small,
    keywordstyle=\color{blue},
    commentstyle=\color{green},
    stringstyle=\color{red},
    identifierstyle=\color{black},
    morekeywords={uint8_t, uint16_t, uint32_t},
    showstringspaces=false,
    tabsize=4,
    frame=single,
    numbers=left,
    numberstyle=\tiny,
    numbersep=5pt,
}

%Konec preambule


\begin{document}

\begin{titlepage}
    \begin{center}
        
        {\Huge \textsc{Vysoké učení technické v Brně}\\[0.5em]}
        {\huge \textsc{Fakulta informačních technologií}}
        \vspace{\stretch{0.382}}
    
        {\LARGE Síťové aplikace\\[0.4em]
        NetFlow v5 exportér}
    
        \vspace{\stretch{0.618}}
        {\Large \hfill {Machala Roman (xmacha86)}}
    \end{center}

\end{titlepage}
%Titulni strana

\tableofcontents
\pagebreak

\section{Úvod}
    NetFlow je síťový protokol vyvinut společností CISCO, který slouží k monitorování a analýze síťového provozu.
    Umožňuje zachytávat jednotlivé pakety, které následně rozděluje do toků na základě např. zdrojové a cílové IP adresy.
    Tyto data jsou zachycena kolektorem a následně pomocí protokolu UDP jsou zaslána do centrálního sběrače (kolektoru), kde
    mohou být analyzována za účelem zjistit vytížení síte, anomálie nebo napříkald i bezpečnostních incidentů\cite{Cisco_2019}.

    Pro protokol NetFlow existuje mnoho verzí, nicméně tato práce se 
    zabývá pouze verzí v5, což je jedna z nejpoužívanějších verzí vůbec dostupná na drtivé většině
    routerů různých výrobců.Tato verze byla navržena pro sběr informací síťového
    provozu pouze na bázi IPv4. Pro podporu IPv6 je nutno vuyžívat vyšších verzí, jako je například NetFlow v9\cite{ManageEngine}.

\subsection{IP Tok}
    Jak již bylo zmíněno výše, protokol NetFlow pracuje s jednotlivými paketami, které rozděluje do \textbf{toků}.
    Každý takový tok je určen unikátní kombinací následujících informací:
    \begin{itemize}
        \item{Zdrojová IP adresa}
        \item{Cílová IP adresa}
        \item{Zdrojový PORT}
        \item{Cílový PORT}
        \item{Typ služby (ToS)}
        \item{Protokol 3 vrstvy}
        \item{Rozhranní}
    \end{itemize}

    Všechny toky jsou pouze jednosměrné, proto například prohozením adres odesílatele a příjemce nevznikne
    nový tok, jak je demonstrováno na obrázku \ref{priklad1}.
    \vspace{2cm}

    \begin{figure}[h!]
        \begin{minipage}[t]{0.45\textwidth}
            \subsection*{Tok 1:}
            \begin{tabbing}
                \hspace*{4cm}\=\hspace*{4cm}\= \kill
                Zdrojová IP: \> 192.168.1.1 \\
                Cílová IP: \> 10.0.0.1 \\
                Zdrojový port: \> 12345 \\
                Cílový port: \> 80 \\
                Typ služby (ToS): \> 0 \\
                Protokol 3 vrstvy: \> TCP \\
                Rozhraní: \> eth0 \\
            \end{tabbing}
        \end{minipage}
            \hfill
        \begin{minipage}[t]{0.45\textwidth}
            \subsection*{Tok 2:}
            \begin{tabbing}
                \hspace*{4cm}\=\hspace*{4cm}\= \kill
                Zdrojová IP: \> 10.0.0.1 \\
                Cílová IP: \> 192.168.1.1 \\
                Zdrojový port: \> 80 \\
                Cílový port: \> 12345 \\
                Typ služby (ToS): \> 0 \\
                Protokol 3 vrstvy: \> TCP \\
                Rozhraní: \> eth0 \\
            \end{tabbing}
        \end{minipage}
        \caption{Příklad 2 různých toků}
        \label{priklad1}
    \end{figure}

    Z obrázku \ref{priklad1} může být taky patrné, že protokol nebere ohled na obsah jednotlivých paket, co se týče rozdělení paket do toků.
    V rámci protokolu NetFlow v5 se s obsahem paket nijak nemanipuluje, uchovávají se ovšem další informace, které jsou vhodné pro 
    statistické vyobrazení a monitorování síťového provozu.
    
    \subsection{Struktura toků/NetFlow v5 datagram}\label{Datagram}
    Kromě klíčových informací sloužících k rozdělení jednotlivých paket do toků se ukládají i jiné informace o každém toku.
    Výčet těchto informací je vyobrazen jako struktura v jazyce C na obrázku \ref{netflowstruct}. V tomto formátu jsou jednotlivé toky
    pomocí UDP protokolu odeslány na kolektor. 
    
    Jelikož je používán protokol UDP pro export datagramů, je možné, že některé datagramy mohou být 
    ztraceny. Z tohoto důvodu novější verze tohoto protokolu, přesněji verze 5, 7 a 8 obsahuje hlavička, dále popsána na obrázku \ref{netflowhdr}, kontrolní číslo toku (flow control number), které je 
    rovno kontrolnímu číslu předešlého toku + počet toků v předešlém datagramu. Při přijetí nového datagramu tak kolektor může zkontrolovat, zda došlo 
    ke ztrátě toků\cite{Cisco_datagram}\cite{IBM_header}.

    \begin{figure}[h!]
        \centering
        \begin{lstlisting}[style=CStyle]
            typedef struct NetFlowv5 {
                uint32_t srcaddr;       /* Source IP address */
                uint32_t dstaddr;       /* Destination IP address */
                uint32_t nexthop;       /* IP address of next hop router */
                uint16_t input;         /* SNMP index of input interface */
                uint16_t output;        /* SNMP index of output interface */
                uint32_t dPkts;         /* Packets in the flow */
                uint32_t dOctets;       /* Total number of Layer 3 bytes */
                uint32_t first;         /* SysUptime at start of flow */
                uint32_t last;          /* SysUptime at the time the last packet */
                uint16_t srcport;       /* TCP/UDP source port number */
                uint16_t dstport;       /* TCP/UDP destination port number */
                uint8_t pad1;           /* Unused (zero) bytes */
                uint8_t tcp_flags;      /* Cumulative OR of TCP flags */
                uint8_t prot;           /* IP protocol type (for example, TCP = 6) */
                uint8_t tos;            /* IP type of service (ToS) */
                uint16_t src_as;        /* Autonomous system number of the source */
                uint16_t dst_as;        /* Autonomous system number of the dest */
                uint8_t src_mask;       /* Source address prefix mask bits */
                uint8_t dst_mask;       /* Destination address prefix mask bits */
                uint16_t pad2;          /* Unused (zero) bytes */
            } netflowv5;
        \end{lstlisting}
        \caption{NetFlow v5 tělo datagramu}
        \label{netflowstruct}
    \end{figure}

    \begin{figure}[h!]
        \centering
        \begin{lstlisting}[style=CStyle]
            typedef struct NetFlowHeader {
                uint16_t version;            /* NetFlow export format version num */
                uint16_t count;              /* Number of exported flows */
                uint32_t sysUptime;          /* Current time in ms since start */
                uint32_t unix_secs;          /* Current count of seconds since CUT */
                uint32_t unix_nsecs;         /* Residual nanoseconds since CUT */
                uint32_t flow_sequence;      /* Sequence cnt of total flows seen */
                uint8_t engine_type;         /* Type of flow-switching engine */
                uint8_t engine_id;           /* Slot num of flow-switching engine */
                uint16_t sampling_interval;  /* Sampling mode + interval (2b - 14b) */
            } NetFlowHeader;
        \end{lstlisting}
        \caption{NetFlow v5 hlavička datagramu}
        \label{netflowhdr}
    \end{figure}

    \pagebreak
    \subsection{Rozdíly v implementaci}
    V této sekci jsou popsány limitace procesu návrhu a implementace exportéru pro protokol NetFlow v5. Tento exportér 
    zpracovává pakety z poskytnutého PCAP souboru. Z tohoto důvodu se s implementací řeší i různé omezení. Následující
    výčet položek struktury datagramu, popsanou v \ref{Datagram}, jsou z důvodu načítání paket z PCAP souboru nedostupné:

    \begin{enumerate}
        \item{IP adresa routeru dalšího skoku}
        \item{SNMP index vstupního Rozhranní}
        \item{SNMP index výstupního Rozhranní}
        \item{SysUptime od začátku toku - doba od spuštění zařízení na kterém byly zachycenz pakety}
        \item{SysUptime na konci přijetí posledního paketu - doba od spuštění zařízení až po zachycení poslední pakety}
        \item{Číslo autonomního systému zdroje}
        \item{Číslo autonomního systému cíle}
        \item{Prefix masky zdrojové adresy}
        \item{Prefix masky cílové adresy}
    \end{enumerate}

    Kromě tohoto se implementovaný exportér zaměřuje pouze na analýzu TCP paketů. Jednotlivé detaily jsou přiblíženy u samotné implementace.


    \section{Návrh aplikace}
    \subsection{Logické rozdělení aplikace}

    \section{Přiblíženní implementace}
    \subsection{Zpracování argumentů}
    \subsection{Hashovací tabulka}
    \subsection{Logika exportéru}

    \section{Testování aplikace}
    \subsection{Návrh testů}
    \subsection{Referenční programy a jejich použití}
    \subsection{Vyhodnocení testování}

    \section{Návod na použití}

    \section{Závěr}

    \pagebreak

    \bibliographystyle{bib-styles/Pysny/czplain}
    \bibliography{manual}

\end{document}